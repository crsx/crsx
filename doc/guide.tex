%% $Id: guide.tex,v 3.6 2013/09/29 05:49:57 krisrose Exp $
\documentclass[11pt]{article} %style: font size.
%% $Id: setup.tex,v 3.1 2013/08/23 07:12:32 krisrose Exp $

%% Format.
\usepackage[utf8]{inputenc}
\usepackage{amsmath,amssymb,stmaryrd,textcomp}
\allowdisplaybreaks %style
\usepackage[normalem]{ulem}
\usepackage{utf8math}
\usepackage{url}
\usepackage{xspace}
\usepackage{rcs}
\usepackage[all]{xy}

%% Macros.
\newcommand{\HAX}{\text{HACS}\xspace}
\newcommand{\CRSX}{\text{CRSX}\xspace}
\newcommand{\ie}{\textit{i.e.}\xspace}
\newcommand{\eg}{\textit{e.g.}\xspace}
\newcommand{\etc}{\textit{etc.}\xspace}
\newcommand{\aka}{\textit{aka.}\xspace}
\newcommand{\TBD}[1][]{\textit{To Be Done…#1}\marginpar{\bf TBD}}

%% Units
\usepackage{amsthm}
\swapnumbers %style
\newtheorem{theorem}{Theorem}[subsection]
\newtheorem{lemma}[theorem]{Lemma}
\theoremstyle{definition}
\newtheorem{definition}[theorem]{Definition}
\newtheorem{proposition}[theorem]{Proposition}
\newtheorem{example}[theorem]{Example}
\newtheorem{notation}[theorem]{Notation}
\newtheorem{invariant}[theorem]{Invariant}
\theoremstyle{remark}
\newtheorem{remark}[theorem]{Remark}

%% Code listings.
\usepackage{fancyvrb}
\CustomVerbatimEnvironment{code}{Verbatim}{tabsize=4,fontsize=\small,numbers=left,numberblanklines=false,xleftmargin=\parindent}
\CustomVerbatimCommand{\inputcode}{VerbatimInput}{tabsize=4,fontsize=\footnotesize,numbers=left,numberblanklines=false,xleftmargin=\parindent}
\DefineShortVerb{\"} % makes " work in verbatim even when \active

%% Identifiers
\newcommand{\set}[1]{\textit{#1}}
\newcommand{\op}[1]{\operatorname{#1}}

\def\version$#1${\thanks{\textbf{DRAFT---Feedback appreciated!} CVS #1.}}
%% Style.
\usepackage[margin=1in]{geometry}
\usepackage[T1]{fontenc}
\usepackage{charter}
\SelectTips{eu}{}
\bibliographystyle{plainurl}

%% Topmatter.
\title{Higher-order Attribute Contraction Schemes\\\emph{User's Guide}%
  \version$Id: guide.tex,v 3.6 2013/09/29 05:49:57 krisrose Exp $}
\author{Kristoffer H. Rose}
\RCSdate$Date: 2013/09/29 05:49:57 $

\begin{document}
\maketitle

\begin{abstract}\noindent
  \HAX is a language for programming compilers.  This guide introduces the basic features of \HAX.
  \HAX is distributed as part of the \CRSX system~\cite{crsx}.
%%%
%  \HAX is a front-end for the \CRSX system\footnote{\CRSX and \HAX (with this document) are
%    available from \url{http://crsx.sf.net}.} specifically designed for programming \emph{compilers}
%  and similar transformation systems. \HAX is distinguished by combining several advanced idioms:
%  \begin{itemize}
%  \item Integration with parser generators.
%  \item Full higher-order abstract syntax.
%  \item Transformations are expressed in native syntax.
%  \item Attribute grammars.
%  \item Inference rules.
%  \item TODO: constraint solvers and hashing.
%  \end{itemize}
%  This guide presents a complete overview of how \HAX can be used to build and run a simple
%  compiler.
\end{abstract}

%\begin{figure}[t]\footnotesize
\tableofcontents
%\end{figure}
\clearpage

%%%%%%%%%%%%%%%%%%%%%%%%%%%%%%%%%%%%%%%%%%%%%%%%%%%%%%%%%%%%%%%%%%%%%%%%

\section{Introduction}

With \HAX it is possible to create a fully functional compiler by writing a single source file.

\subsection{Running the example}

Get the dragon


 intended as a fully formalized yet pedagogical platform
for implementing systems for ``semantics-based program transformations'' such as compilers and
optimizers.

\subsection{A Taste}

For a taste of the system, here is a small example file~\cite[samples/guide/fupz.hx]{crsx}.
%%
\inputcode{../samples/guide/fupz.hx}
%%
The example illustrates these points:
%%
\begin{itemize}
\item Comments are possible.
\item A specification is always a module using the ubiquitous "{}"s.
\item The specification is structured into ``sorts'' linked to grammar productions.
\item Full Unicode can be used in general, and in particular concrete syntax is in "⟦…⟧" and grammar
  recursion in~"⟨…⟩".
\item The "⟦⟨[V]⟩ → ⟨E[V]⟩⟧" is special syntax for declaring higher order syntax, which will be
  explained.\footnote{The special syntax for abstractions is explained in Section~\ref{sec:hos}.}
\item Tokens are declared with regular expressions.
\item Transformations are called ``schemes'' and also have concrete syntax.
\item Schemes are specified with \emph{rules} of the form ``pattern'' → ``result,'' also using
  syntax.
\item The last declaration illustrates how a rule can match higher order syntax and even perform
  substitution of bound variables.
\end{itemize}
%%

\subsection{First Run}



\subsection{Overview}

In the rest of this document we introduce the most important features of \HAX specifications.

\paragraph*{Syntax.} Used to specify the external syntax of all representations used in files and
rules. Structured around \emph{productions} (\aka \emph{syntactic sorts}) with additional notation
for \emph{concrete syntax sugar} so a parser can be built to map from text into abstract syntax, and
\emph{syntax formatting instructions} so a pretty-printer can be built to map from each used
abstract syntax production back to text.  The syntax supported by \HAX is peculiar in that it can
construct uses \emph{higher order data}, which means that it supports parsing that includes variable
``binders'' and ``occurrences'' natively.  Section~\ref{sec:syntax} below explains the basic syntax
mechanism, and further details are given in the reference Appendix~\ref{app:syntax}.

\paragraph*{Translation Schemes.} Base rules for how terms of the (abstract) syntax are transformed
between the used intermediate representations, possibly depending on attribute annotations resulting
from analysis.  Typical translation schemes in a compiler are ``normalization'' and ``code
generation.''  Section~\ref{sec:translation} explains this, again with support in the reference
Appendix~\ref{app:translation}.

\paragraph*{Attribution Rules.} Rules for how attributes are assigned, in various forms, including
\emph{attribute grammars}, \emph{deterministic inference rules}, and \emph{tables}.  Attributes
typically are a combination of abstract syntax and values of \emph{semantic sorts}; an example
attribution would be ``type assignment,'' which assigns a type attribute to each subexpression.
Section~\ref{sec:attribution} explains this with support in the reference
Appendix~\ref{app:attribution}.

\paragraph*{Optimization Equivalences.} Finally, an experimental feature is to indicate the
preferred form of semantically equivalent expressions, which can then be used to modify the
translation schemes to be optimizing in a safe way.  This is documented by
Section~\ref{sec:attribution} and Appendix~\ref{app:attribution}.




\section{Syntax}
\label{sec:syntax}

\HAX works with \emph{sorted} abstract syntax trees (AST).  Specifically, \HAX provides a special
notation for specifying abstract and concrete syntax together with the used sorts corresponding to
the abstract syntax productions (and thus the AST is what is called a \emph{many-sorted algebraic
  term}).  In this section we explain the common syntax of the combined sort and grammar
specifications.

\subsection{First Order Syntax}

We start off with the basic (first order) notational conventions.  (The list is rather long but
readers familiar with \emph{parser generation} will hopefully find most of this section
unsurprising, and we proceed with an example right afterwards.)

\begin{notation}\label{nota:syntax}
  %%
  \HAX uses the following notations for specifying the syntax to use for terms.\footnote{The
    notations use special Unicode characters: see the grammar in the appendix for their codes and
    alternatives.}
  %%
  \begin{itemize}

  \item \HAX \emph{production names} are capitalized, so we can for example use "Exp" for the
    production of expressions.  The name of a production also serves as the name of its \emph{sort},
    \ie, the semantic category that is used internally for abstract syntax trees with that root
    production.  If particular instances of a sort need to be referenced later they can be
    \emph{disambiguated} with an "#"$i$ suffix, \eg, "Exp#2", where $i$ is an optional number or
    other label.

  \item A sort is declared by "sort" declarations of the name optionally followed by a number of
    \emph{abstract syntax production} alternatives, each starting with a~"|".

  \item Double square brackets "⟦…⟧" are used for \emph{concrete syntax} but can contain nested
    angle brackets "⟨…⟩" with \emph{production references} like "⟨Exp⟩" for an expression (as well
    as several other things that we will come to later).  We for example write "⟦⟨Exp⟩+⟨Exp⟩⟧" to
    describe the form where two expressions are separated by a "+" sign.

  \item A trailing "@"$p$ for some precedence integer $p$ indicates that either the subexpression or
    the entire alternative (as appropriate) should be considered to have the indicated precedence,
    with higher numbers indicating higher precedence, \ie, tighter association.  (For details on the
    limitations of how the precedence and left recursion mechanisms are implemented, see
    Appendix~\ref{app:prec}.)

  \item "sugar ⟦…⟧→…" alternatives specify equivalent forms for existing syntax: anything matching
    the left alternative will be interpreted the same as the right one (which must have been
    previously defined); references must be disambiguated.

  \item "token" declarations define production of a single \emph{string token} specified by a
    regular expression; similarly "space" defines the regular expression for the special token that
    is permitted and ignored between other tokens. (The token and regular expression formats are
    quite extensive and detailed in Appendix~\ref{app:token-reference}, including how one deals with
    lookahead, keywords, special modes, nested comments, \etc)

  \item \HAX supports C/Java-style comments, either with "//" for single line comments or "/*"…"*/" for
    delimited comments (may be multi-line and nested).

  \end{itemize}
\end{notation}

With these we can handle much of traditional parser generation.

\begin{example}\label{ex:deriv1}
  %%
  Numeric expressions are often specified with an informal syntax description like the
  following. First a \emph{abstract syntax} in formal (non-\HAX) notation:
  %%
  \begin{align}
    \tag{Exp}
    e &::= e+e \mid e-e \mid e×e \mid \frac{e}{e} \mid fe \mid n \\
    \tag{Fun}
    f &::= \op{sin} \mid \op{cos} \mid \op{ln} \mid \op{exp}
  \end{align}
  %%
  $n$ ranges over non-negative number literals (that we are not particularly interested in here
  except that they are expected to include $0$ and $1$).  In addition, we suppose standard
  disambiguation of nested expressions by the conventions that
  %%
  \begin{itemize}

  \item parentheses can be used, making $(e)$ a valid expression;

  \item parenthesis have highest precedence, and function application has higher precedence than
    $×$, which in turn has higher precedence than $+$ and~$-$ (no precedence is given for fraction
    expressions as the notation takes care of that);

  \item a sequence of operators with the same precedence associates to the left; and

  \item in a $\pm$-sequence one can omit a leading~$0$.

  \end{itemize}
  %%
  So $-1+2-3×\frac{4}{\op{ln}5}$ is the same expression as $((0-1)+2)-(3×(\frac{4}{\op{ln}5}))$.

  We can use a very similar grammar as a \HAX specification, except we limit ourselves to the more
  usual textual "/" for fractions, and choose a specific precedence numbering.
  %%
\begin{code}
sort Exp 
| ⟦ ⟨Exp@1⟩ + ⟨Exp@2⟩ ⟧@1                  
| ⟦ ⟨Exp@1⟩ - ⟨Exp@2⟩ ⟧@1                           
| ⟦ ⟨Exp@2⟩ * ⟨Exp@3⟩ ⟧@2                           
| ⟦ ⟨Exp@2⟩ / ⟨Exp@3⟩ ⟧@2                           
| ⟦ ⟨Fun⟩ ⟨Exp@4⟩ ⟧@3                               
| ⟦ ⟨Int⟩ ⟧@4                                       
                                           
| sugar ⟦ ( ⟨Exp#⟩ ) ⟧@4 → ⟦ ⟨Exp#⟩ ⟧               
| sugar ⟦ + ⟨Exp#@2⟩ ⟧@1 → ⟦ 0 + ⟨Exp#⟩ ⟧           
| sugar ⟦ - ⟨Exp#@2⟩ ⟧@1 → ⟦ 0 - ⟨Exp#⟩ ⟧
;         

sort Fun
| ⟦sin⟧@2 | ⟦cos⟧@2 | ⟦ln⟧@2 | ⟦exp⟧@2 ;

token Int | [0-9]+ ;
\end{code}
  %%
  The effect of the declaration is that we achieve the capability to parse expressions with
  parentheses and all the variations of the original syntax yet the resulting abstract syntax
  behaves as if it had been specified with just
  %%
\begin{code}[numbers=none]
sort Exp|⟦⟨Exp⟩+⟨Exp⟩⟧|⟦⟨Exp⟩-⟨Exp⟩⟧
|⟦⟨Exp⟩*⟨Exp⟩⟧|⟦⟨Exp⟩/⟨Exp⟩⟧|⟦⟨Fun⟩⟨Exp⟩⟧|⟦⟨Int⟩⟧;
\end{code}
  %%
  which are then all the "Exp" patterns that we need to be concerned with instantiating for
  compilation schemes and attributions below.

  Our example expression from before can be written as "-1+2-3*(4/ln 5)" and will be parsed by \HAX
  into an abstract syntax tree like…\TBD

  Finally, we remark that we have given the predefined functions precedence "@1" so we can add
  function expressions later.
  %%
\end{example}

\subsection{Higher Order Syntax}
\label{sec:hos}

We also need a way to express functional expressions, which require the higher order aspects of
\HAX, and is our first extension of the "⟨⟩"-notation from above.  This is used for managing
\emph{binding} (also known as \emph{scope management tables}), which comes from the traditions of
\emph{λ-calculus} and \emph{higher-order abstract syntax}.

\begin{notation}\label{nota:binder-sorts}
  %%
  \HAX has special support for parsing bindings and bound occurrences of variables. These
  \emph{will} be part of the abstract syntax in the form of \emph{higher order data}, \ie, data with
  embedded bound variables.
  %%
  \begin{itemize}

  \item The notation "⟨[Var]⟩" means that a "Var" token is parsed and introduced as a
    \emph{binder} for some scope (to be specified separately);

  \item "⟨Exp[Var]⟩" specifies that the subexpression generated by "⟨Exp⟩" is in fact the
    \emph{scope} of the binder variable named by the "Var" token (which must have exactly one
    binding specified in the alternative;\footnote{Currently \HAX imposes the limitation that the
      binding specification must be to the left of the scope.} use the "Sort#"$i$ form of token
    sorts if disambiguation is necessary); and \TBD[Consider using \texttt{[Var]Exp} or
    \texttt{Var.Exp}.]

  \item Just using "⟨Var⟩" (as before) for a token that has binders means that the "Var" token will
    be identified with any bound tokens with the same text, if such exist for which it is in scope,
    and will then correspond to an \emph{occurrence} of the variable.

  \end{itemize}
  %%
\end{notation}

\begin{example}\label{ex:deriv2}
  %%
  We extend the numeric functional expressions of Example~\ref{ex:deriv1} with a symbolic notation for
  functions:
  %%
  \begin{align}
    \tag{Exp-2}
    e &::= … \mid x \\
    \tag{Fun-2}
    f &::= … \mid [x↦e]
  \end{align}
  %%
  where $x$ should be a \emph{variable} (lower case letter), which is then \emph{bound} inside the
  following $e$ expression in the usual way.  We add this to the \HAX code with the additional
  declarations
  %%
\begin{code}
sort Exp | ⟦ ⟨Var⟩ ⟧@4 ;
sort Fun | ⟦ [ ⟨[Var#1]⟩ ↦ ⟨Exp[Var#1]⟩ ] ⟧ ;
token Var | [a-z] ;
\end{code}
  %%
  Specifically, "⟦[⟨[Var#1]⟩↦⟨Exp[Var#1]⟩]⟧" should be read as follows:
  %%
  \begin{enumerate}

  \item A concrete "[" character.

  \item\label{item:x} A "Var" token, which is introduced as a \emph{binder}, indicated by the
    surrounding "[]" and disambiguated for later by adding "#1".

  \item A concrete "↦" mapping arrow (unicode code point "\u21a6")

  \item An "Exp" expression which is the \emph{scope} for the "Var" binder from before, \ie, wherein
    all occurrences of "Var" tokens identical to the one in the trailing "[]"s (namely "Var#1" from
    item \ref{item:x}) are considered \emph{bound occurrences}.

  \item A concrete "]" character.

  \end{enumerate}
  %%
  The example also illustrates that a sort can have many separate declarations that all contribute
  to the definition of the sort: the declarations here add to the ones from Example~\ref{ex:deriv1}.
  %%
\end{example}

\TBD[Example with Kleene star/plus.]


\section{Translation Schemes}
\label{sec:translation}

Once we have one or more abstract syntaxes to work with, we are ready for specifying \emph{schemes}
for how the source abstract syntax translates to the target abstract syntax.

\subsection{Declaration}

Schemes can be specified either using custom syntax or using the built-in term syntax. In its
simplest form, a scheme is specified by simply declaring the syntax form of it with the keyword
"scheme", and then give a number of \emph{rewrite rules} that define what the replacement
should be for each variation of the scheme.

\begin{notation}\label{nota:scheme}
  A \emph{scheme} is just a syntactic form marked with the prefix "scheme" that is permitted to have
  rewrite rules that define its behaviour, typically what instances of the scheme should be
  transformed to.
\end{notation}


\begin{example}\label{ex:deriv3}
  %%
  The \emph{derivative} (in Lagrange notation) of a function $f$ is written $f'$. We can declare
  this as a scheme with
  %%
\begin{code}
sort Fun | scheme ⟦⟨Fun@2⟩'⟧@2 ;
\end{code}
  %%
  The precedence is to explicitly allow left recursion, so we can write "sin''".
  (In this and several following examples in this section, we are essentially encoding a higher
  order rewrite system for symbolic differentiation due to Knuth \cite[p.337]{Knuth:1973:TAOCP} as a
  \HAX specification over the numeric expressions from Example \ref{ex:deriv1}
  and~\ref{ex:deriv2}.)
  %%
\end{example}

Once we have a declared scheme then we need to populate it with transformation rules. This can be
done in a number of ways.

\subsection{Rewriting Rules}

The principal way of declaring a scheme is with rewrite rules.

\begin{notation}\label{nota:rule}
  A \emph{rewrite rule} has the following structure:
  %%
  \begin{itemize}

  \item A basic rule consists of a \emph{pattern} that is a fully disambiguated instance of a scheme
    (to the left), then an arrow (→), and finally a \emph{contraction} (to the right), which must be
    a term of the same sort and where all inserted references must refer to disambiguated references
    in the pattern. The intention is that the rules for a scheme fully explain how the defined form
    is eliminated.

  \item Rules sometimes have a \emph{rule prefix} of the form \emph{name}"("\emph{options}"):" in
    front of the rule; we will show a few examples of this along the way.

%  \item If angle brackets "⟨⟩" contain something other than a disambiguated production reference
%    like "⟨Exp#1⟩", then it must be a \emph{term}, which starts with a constructor (defined or
%    data, as we shall see later) followed by a sequence of parameters in "()"s.

  \end{itemize}
  %%
\end{notation}

Before we continue with our higher order example, here is a simple first order one.

\begin{example}\label{ex:bool}
  %%
  We will be computing with logic, so here are some notation and simplification schemes for
  booleans.
  %%
\begin{code}
sort B
| ⟦ t ⟧@4 | ⟦ f ⟧@4 | sugar ⟦ (⟨B#⟩) ⟧@4→B#

| scheme ⟦ ⟨B@2⟩ ∨ ⟨B@1⟩ ⟧@1;
⟦t∨⟨B⟩⟧ → ⟦t⟧;  ⟦f∨⟨B#⟩⟧ → B#;

| scheme ⟦ ⟨B@3⟩ ∧ ⟨B@2⟩ ⟧@2;
⟦t∧⟨B#⟩⟧ → B#;  ⟦f∧⟨B⟩⟧ → ⟦f⟧;

| scheme ⟦ ¬ ⟨B@3⟩ ⟧@3;
⟦¬t⟧ → ⟦f⟧;  ⟦¬f⟧ → ⟦t⟧;
\end{code}
  %%
  The example combines all the first order features we have seen: data values (for true and false),
  syntactic sugar (parenthesis), and defined schemes with rewrite rules (boolean connectives).
  %%
\end{example}

\begin{example}\label{ex:deriv4}
  %%
  The \emph{derivative} (in Lagrange notation) of a function $f$ is written $f'$. The following
  rewrite rules define the derivative as a scheme with rules for how each the function forms of
  Example~\ref{ex:deriv1} and \ref{ex:deriv2} transform into another form that corresponds to the
  derivative. Each rule handles an instance of the scheme for each of the possible instantiation of
  the "Fun" subterm.
  %%
  We first declare that we will now (continue to) populate the "Fun" sort.
  %%
\begin{code}
sort Fun ;
\end{code}
  %%
  We then define the transform for each possible form of the scheme; it is important that each rule
  pattern is strictly consistent with the grammar rules: each left hand side is an instance of the
  defined scheme "⟦⟨Fun⟩'⟧" and each right hand side a value of sort "Fun"; in addition the rules
  fully cover all forms of the scheme.
  %%
\begin{code}  
⟦sin'⟧ → ⟦cos⟧ ;
⟦cos'⟧ → ⟦[a ↦ - sin a]⟧ ;
⟦ln'⟧  → ⟦[z ↦ 1/z]⟧ ;
⟦exp'⟧ → ⟦exp⟧ ;

⟦[x ↦ ⟨Exp#[x]⟩]'⟧ → ⟦[y ↦ D(y)[z↦⟨Exp#[z]⟩]]⟧ ;
\end{code}
  %%
  Notice how "cos" and "ln" have function expressions as their derivative, \eg, for "cos" the
  derivative "⟦[z↦1/z]⟧" denotes the numeric function with a bound variable identified by the "Var"
  token "z" and an "Exp" subexpression which is the division~"1/z".  When a bound variable like "z"
  is introduced in this way, \HAX ensures that it is suitably named to be different from all other
  variables in scope.  The last rule, the derivative of a function expression "[x↦⟨Exp[x]⟩]", is
  defined in terms of a ``dependent derivative'' and will be explained with Example~\ref{ex:deriv5}
  below.
  %%
\end{example}

\begin{example}\label{ex:deriv5}
  %%
  
  In Euler notation the derivative of an expression with respect to a dependent variable is written
  $D_xf$. We can represent this in \HAX as follows:
  %%
\begin{code}
sort Exp | scheme ⟦ D ⟨Exp⟩ [⟨[Var#1]⟩↦⟨Exp[Var#1]⟩] ⟧ ;
\end{code}
  %%
  where the "D"-construction has two arguments: the dependent variable (of sort "Exp") from the
  context, in "()"s, and the function expression.
%
%  Notice that even
%  though there are no "⟨⟩"s in this declaration, everything is using the built-in syntax, and the
%  sorts are specified with the rules from Notation~\ref{nota:binder-sorts} (without the "⟨⟩"s). In
%  particular we do not have to specify \emph{where} the variable is bound as it will always be bound
%  with a binding right on the subscope of the second parameter of "D".
%
  Now for the derivative of each expression form.  First constants and variables.
  %%
\begin{code}[firstnumber=last]
⟦ D⟨Exp#⟩[x↦⟨Int#⟩] ⟧ → ⟦0⟧ ;

Bound:          ⟦ D⟨Exp#⟩[x↦x] ⟧ → ⟦1⟧ ;
Indep(Free(y)): ⟦ D⟨Exp#⟩[x↦y] ⟧ → ⟦0⟧ ;
\end{code}
  %%
  The last two rules have an optional rule prefix, used to identify the rule and, for the last,
  declare with an option that a free variable is intended in the rule.
  The patterns of the rules are almost identical, different only on the form of the scope of~"x";
  this is a common pattern. Specifically, there are two rules for variables: one matching an
  instance of the dependent variable and one for a variable that is different, which is ensured with
  our first example of an option, the "Free(y)" indicator, which restricts "y" to variables that are
  not bound in the pattern, thus in particular not~"x".

  We continue with the usual binary operations.
  %%
\begin{code}[firstnumber=last]
⟦ D⟨Exp#⟩[x↦⟨Exp#1[x]⟩+⟨Exp#2[x]⟩] ⟧ → ⟦D⟨Exp#⟩[y↦⟨Exp#1[y]⟩] + D⟨Exp#⟩[z↦⟨Exp#2[z]⟩]⟧ ;
⟦ D⟨Exp#⟩[x↦⟨Exp#1[x]⟩-⟨Exp#2[x]⟩] ⟧ → ⟦D⟨Exp#⟩[y↦⟨Exp#1[y]⟩] - D⟨Exp#⟩[z↦⟨Exp#2[z]⟩]⟧ ;
\end{code}
  %%
  These illustrate one way bound variables are managed by \HAX rules. Specifically, the pattern
  (left hand side) fragment "x↦⟨Exp#1[x]⟩+⟨Exp#2[x]⟩" specifies that the pattern will match scopes
  that bind some variable, which we shall refer to as "x", and that the scope is a "+"-expression
  with two sub-"Exp" terms that can both contain the variable. In the contraction (right hand side)
  we then \emph{split} the scope in two. The first scope that is created is "y↦⟨Exp#1[y]⟩", which
  has the new binder "y" and a copy of what matched "Exp#1" before, except all occurrences of "x"
  are replaced by~"y". (In some textbooks such a replacement would have been written "{y/x}Exp#1".)
  This ensures that the scope is split cleanly, with the two new scopes over "y" and "z" completely
  disjoint. Such ``scope splitting'' is immensely useful, and is the reason we separated the
  dependent variable from the context and the bound variable in the scopes.  (In fact this is the
  simplest case of a \emph{substitution} that the following rules will have deeper examples of.)

  The product and division rules are slightly more involved.
  %%
\begin{code}[firstnumber=last]
⟦ D⟨Exp#⟩[x↦⟨Exp#1[x]⟩ * ⟨Exp#2[x]⟩] ⟧
 → ⟦ D⟨Exp#⟩[x↦⟨Exp#1[x]⟩] * ⟨Exp#2[#]⟩ + ⟨Exp#1[#]⟩ * D⟨Exp#⟩[x↦⟨Exp#2[x]⟩] ⟧ ;

⟦ D⟨Exp#⟩[x↦⟨Exp#1[x]⟩ / ⟨Exp#2[x]⟩] ⟧
 → ⟦ ( D⟨Exp#⟩[x↦⟨Exp#1[x]⟩] * ⟨Exp#2[#]⟩ - ⟨Exp#1[#]⟩ * D⟨Exp#⟩[x↦⟨Exp#2[x]⟩] )
      / (⟨Exp#2[#]⟩ * ⟨Exp#2[#]⟩) ⟧ ;
\end{code}
  %%
  In these rules the pattern (left hand side) matches several separate units, "Exp#", "Exp#1[x]",
  and "Exp#2[x]", just like the sum and difference rules. But here the contraction (right hand side)
  combines them in a more complex way than merely creating new scopes. Specifically, we have
  instances of "Exp#1[#]" and "Exp#2[#]" in the contraction. These are \emph{proper substitutions}:
  if the pattern matched "Exp#1[x]" then the contraction "Exp#1[#]" means ``create copy of what
  matched "Exp#1" except substitute all occurrences of the bound variable "x" with a copy of what
  matched ("Exp")"#".''  (Again, a traditional formal way of writing such a substitution would be
  "{#/x}#1".)

  Finally, the chain rule.
  %%
\begin{code}[firstnumber=last]
⟦ D(⟨Exp#⟩)[x↦⟨Fun#⟩⟨Exp#[x]⟩] ⟧ → ⟦ ⟨Fun#⟩' ⟨Exp#[#]⟩ * D(⟨Exp#⟩)[x↦⟨Exp#[x]⟩] ⟧ ;
\end{code}
  %%
  Note how the contraction uses the "'" notation from Example~\ref{ex:deriv4} to differentiate the
  function.  Furthermore, the chain rule illustrates one last point: \emph{omission} of bound
  variables: the pattern contains "x ↦ ⟨#Fun⟩ ⟨Exp#[x]⟩", where both "Fun#" and "Exp#[x]" are in the
  scope of "x" yet only "Exp#[x]" has a marker indicating to keep track of references to the "x"
  bound variable.  This is significant, and indicates that the pattern only matches if "#Fun"
  \emph{does not contain instances of}~"x". Formally, "#Fun" is said to be \emph{weak} for "x", and
  this has the advantage that in the contraction we can use "#Fun" safely outside of any scope, as
  the chain rule indeed needs.
  %%
\end{example}

We can now rewrite…\TBD



\section{Attributions}
\label{sec:attribution}

In addition to translation schemes that traverse and create abstract syntax trees, compilers also
make use of analysis to enrich the abstract syntax tree with \emph{attributes} that are used to
assist the translation schemes.

\subsection{Adding Attributes to Terms}

Attributes come in two flavors: \emph{inherited} attributes that are used to attach additional
information to schemes, and \emph{synthesized} attributes that represent properties of data.  (If
you are familiar with \emph{attribute grammars} then everything in this section should be standard.)

\begin{notation}\label{nota:attributes}
  %% 
  \emph{Attributes} are added to terms using the following notations:
  %% 
  \begin{itemize}

  \item Simple \emph{synthesized attributes} are added after a data term as
    "↑"\emph{name}"("\emph{value}")", where the \emph{name} is a constant identifier and the
    \emph{value} can be any term.

  \item Simple \emph{inherited attributes} are added similarly after a scheme term as
    "↓"\emph{name}"("\emph{value}")".

  \item An \emph{indexed synthesized attribute} is added after a data term as
    "↑"\emph{name}"{"\emph{key}":"\emph{value}"}", where the \emph{key} is either a variable or a
    constant.

  \item An \emph{indexed inherited attribute} is added to a scheme term as
    "↓"\emph{name}"{"\emph{key}":"\emph{value}"}".

  \end{itemize}
\end{notation}

\subsection{Declaring Attributes}

Note that whent using attributes inside "⟦⟧" brackets then they must be inside the unparsed
fragments, \ie, inside "⟨⟩"s.

\begin{notation}[attribute sorts]\label{nota:attribute-sorts}
  Atributes can be declared with special declarations like these after a usual \verb"sort HostSort"
  declaration:
  \begin{itemize}

  \item "| ↑Name(ValueSort)" defines that "HostSort" has the simple synthesized
    attribute "Name" which has "ValueSort" values.

  \item "| ↓Name(ValueSort)" similarly for an simple inherited attribute.

  \item "| ↑Name{KeySort:ValueSort}" defines that "HostSort" has the indexed
    synthesized attribute "Name" which for each constant or variable of "KeySort" has a distinct
    "ValueSort" value.

  \item "| ↓Name{KeySort:ValueSort}" similarly for an indexed inherited attribute.

  \end{itemize}
\end{notation}

\subsection{Computing Attributes}

\begin{example}\label{ex:total}
  Consider our expressions from Examples~\ref{ex:deriv1}, \ref{ex:deriv2}, \ref{ex:deriv3},
  \ref{ex:deriv4}, and \ref{ex:deriv5}; in addition consider the booleans of Example~\ref{ex:bool}
  loaded.  One piece of information that may be of interest for the derivative is to analyze whether
  an expression is always defined, \ie, defined for all possible values of all variables. We can
  represent this information with a \HAX simple synthesized attribute.
  %%
\begin{code}
sort Exp | ↑ TOTAL(B);
\end{code}
  %%
  The declaration adds the synthesized "TOTAL" attribute to the "Exp#" sort.  We then provide
  information to \HAX on how the attribute is synthesized for all "Exp#" forms. For constants and
  most operations this is simple.
  %%
\begin{code}[firstnumber=last]
⟦⟨Int#⟩⟧ ↑ TOTAL(⟦t⟧) ;
⟦⟨Var#⟩⟧ ↑ TOTAL(⟦t⟧) ;

⟦⟨Exp#1↑TOTAL(B#1)⟩ + ⟨Exp#2↑TOTAL(B#2)⟩⟧ ↑ TOTAL(⟦ ⟨B#1⟩ ∧ ⟨B#2⟩ ⟧) ;
⟦⟨Exp#1↑TOTAL(B#1)⟩ - ⟨Exp#2↑TOTAL(B#2)⟩⟧ ↑ TOTAL(⟦ ⟨B#1⟩ ∧ ⟨B#2⟩ ⟧) ;
⟦⟨Exp#1↑TOTAL(B#1)⟩ * ⟨Exp#2↑TOTAL(B#2)⟩⟧ ↑ TOTAL(⟦ ⟨B#1⟩ ∧ ⟨B#2⟩ ⟧) ;
\end{code}
  %%
  For division this is slightly harder, in that division is only total when the divisor is known to
  never be zero. For this we invent an additional boolean synthesized attribute, "NZ" for ``Never
  Zero,'' to be developed in Example~\ref{ex:nz}.
  %%
\begin{code}[firstnumber=last]
⟦⟨Exp#1↑TOTAL(#1)⟩ / ⟨Exp#2↑TOTAL(#2)↑NZ(#nz⟩⟧ ↑ TOTAL(⟦⟨B#1⟩∧⟨B#2⟩∧⟨B#nz⟩⟧) ;
\end{code}
  %%
  \TBD[Reverse disambiguation…]
  %%
\begin{code}[firstnumber=last]
⟦⟨#Fun⟩ ⟨Exp# ↑TOTAL(#)⟩⟧ ↑ TOTAL(#) ;
\end{code}
  %%
\end{example}

\subsection{Inference Rules}

Many published analyses use some logic inference system with rules for proving the existence of and
typically also deriving the value of some desired property.  \HAX supports a special syntax for
entering such systems…\TBD

\subsection{Tables}

In some cases a table format is preferable; this is supported in \HAX by…\TBD


\section{Data Equivalences}
\label{sec:equiv}

Translation schemes like our derivation example can sometimes use a litle help. One thing that can
happen, for example, is that an expression translates to a constant, which creates situations where
arithmetic simplification is possible. This is supported by \HAX systems through \emph{data
  equivalences}, which are simplifications of \emph{non-scheme} terms, which will be folded into the
defined schemes.

\subsection{Unconstrained Equivalences}

\begin{invariant}
  Equivalences must be equivalent for all translation schemes…
\end{invariant}

\begin{example}
  %%
  Arithmetic simplification.
  %%
\begin{code}
Add0R[Equiv]: ⟦ ⟨Exp#⟩ + 0 ⟧ → Exp# ;
Add0L[Equiv]: ⟦ 0 + ⟨Exp#⟩ ⟧ → Exp# ;

Product0R[Equiv]: ⟦ ⟨Exp#⟩ * 0 ⟧ → ⟦0⟧ ;
Product0L[Equiv]: ⟦ 0 * ⟨Exp#⟩ ⟧ → ⟦0⟧ ;

Product1R[Equiv]: ⟦ ⟨Exp#⟩ * 1 ⟧ → Exp# ;
Product1L[Equiv]: ⟦ 1 * ⟨Exp#⟩ ⟧ → Exp# ;

Minus1[Equiv]:  ⟦ ⟨Exp#⟩ - 0 ⟧ → Exp# ;
Divide1[Equiv]: ⟦ ⟨Exp#⟩ / 1 ⟧ → Exp# ;
\end{code}
  %%
\end{example}

\begin{example}
  %%
  Simplifying created function applications.
  %%
\begin{code}
Beta[Equiv]: ⟦ [x1↦⟨Exp#2.x1⟩] ⟨Exp#1⟩ ⟧ → Exp#2[#1] ;
\end{code}
  %%
\end{example}

\subsection{Constrained Equivalences}

In most cases, however, an equivalence is constrained by an attribute. \TBD


\section{Running \HAX}
\label{sec:run}

\subsection{Limitations}
\begin{itemize}

\item At most one "nested" declaration per "token".

\item It is not possible to use binders and left recursion in the same production (with same
  precedence).

\end{itemize}


\small
\appendix

\section{Grammar Reference}
\label{app:grammar}

We include details about the grammar here.

\subsection{Token Reference}
\label{app:token-reference}

Summary of tokens and regular expressions…\TBD

\begin{code}
sort RE | ⟦ ⟨REChoice*_"|"⟩ ⟧ ;

sort REChoice
| ⟦ nested ⟨RESimple⟩ ⟨RESimple⟩ ⟧
| ⟦ ⟨REUnit*⟩ ⟧
;

sort REUnit | ⟦ ⟨RESimple⟩ ⟨Repeat?⟩ ⟧ ;

sort Repeat | ⟦ ? ⟧
            | ⟦ * ⟧
            | ⟦ * _ ⟨String⟩ ⟧
            | ⟦ + ⟧
            | ⟦ + _ ⟨String⟩ ⟧
            ;

sort RESimple
| ⟦ ⟨String⟩ ⟧
| ⟦ ⟨Word⟩ ⟧
| ⟦ ⟨FragmentRef⟩ ⟧
| ⟦ . ⟧
| ⟦ [ ⟨REClass⟩ ] ⟧
| ⟦ ( ⟨RE⟩ ) ⟧
;

sort REClass
| ⟦ ^ ⟨REClassFirstChar⟩ ⟨RERangeEnd?⟩ ⟨RERange*⟩ ⟧
| ⟦ ⟨REClassFirstChar⟩ ⟨RERangeEnd?⟩ ⟨RERange*⟩ ⟧
;
sort RERange | ⟦ ⟨REClassChar⟩ - ⟨REClassChar⟩ ⟧ ;
sort RERangeEnd | ⟦ - ⟨REClassChar⟩ ⟧ | ;

token REClassFirstChar | [^^\n\\] | ⟨EscapedChar⟩ ;
token REClassChar | [^\]\n\\] | ⟨EscapedChar⟩ ;
\end{code}


\subsection{Precedence and Left Recursion}
\label{app:prec}

Some notes on the internals way \HAX deals with precedence and left recursion.

\begin{remark}[parser precedence]
  %%
  The different precedence levels just become different low level productions, with the exception of
  left recursion, which is eliminated by rewriting rules. So, for example,
  %%
\begin{code}
sort Exp | ⟦ ⟨Exp@1⟩ + ⟨Exp@2⟩ ⟧@1 | ⟦ ⟨Int⟩ ⟧@2 ;
\end{code}
  %%
  really means
  %%
\begin{code}
sort Exp | Exp@1 ;

sort Exp@1 | scheme Leftify[@2, Exp@1Tail]
           | sugar ⟦ Exp@2 Exp@1Tail ⟧ → Leftify[#@2, Exp#@1Tail] ;

sort Exp@1Tail | sugar ⟦+ ⟨Exp#@2⟩⟧ Exp#@1Tail → Plus[#@2, Exp#@1Tail]
               | sugar ⟦⟧ → Done ;

Leftify[#1, Done] → #1 ;
Leftify[#1, Plus[#2, #3]] → Leftify[⟦⟨Exp#1⟩ + ⟨Exp#2⟩⟧, #3] ;

sort Exp@2 | Int ;
\end{code}
  %%
  where "Leftify", "Plus", and "Done", are internal symbols (so probably really named
  \verb"Exp@1Tail@Leftify" \etc). The right to left recursion conversion happens with the "Leftify"
  rules, where the actual left recursive nested expressions are built by the second rule.
  %%
\end{remark}

\subsection{``Raw'' Terms}

\begin{notation}\label{nota:raw}
  %%
  ``Internal'' or ``raw'' term syntax follows these conventions:
  %% 
  \begin{itemize}

  \item Names containing a "#" are \emph{meta-variables}. A meta-variable can be followed by a
    sequence of \emph{meta-arguments} in "[]"s and is then called a \emph{meta-application}.  When
    used with concrete syntax, meta-variables have the form \emph{Sort}"#"\emph{i}.

  \item Names starting with a lower case letter are (plain) \emph{variables}, which can be used as
    \emph{binders} in constructions or as an \emph{occurrence} raw term.

  \item All other names are \emph{constructors}.  A constructor followed by "()"s with a number of
    ","-separated \emph{scoped subterms} is called a \emph{construction}.  The scoped subterms have
    a very specific form: an optional sequence of variables in "[]"s sets up the binders for the
    scoped subterm and then the term itself is last.  Construction is used for both data and defined
    schemes (the latter roughly corresponding to functions).

  \end{itemize}
  %%
\end{notation}

Notice that in raw form we do not have to use meta-variables with the same name as the sort -- "#x"
can be used as the name for the first "Exp"-sorted argument. The binding that is introduced, of "x"
in each rule, is written as a raw "[]"d prefix to the scope subterm. Especially the "Var" rule is
interesting: it shows how the same "x" variable name can be used in the raw form -- the "[x]" -- and
inside parsed content -- the "⟦x⟧" -- the token management knows both are "Var" tokens that must
thus be identified.\footnote{In this case it is a furher lucky coincidence that the lexical form of
  the token has a form that fits the \HAX rules for a raw variable name but this can be repaired
  when necessary, as explained in Appendix~\ref{app:grammar}.}


\subsection{Complete Grammar}

\HAX is a \HAX…

\inputcode{hoacs.hx}


\normalsize
\bibliography{crs}


\end{document}


%% $Log: guide.tex,v $
%% Revision 3.6  2013/09/29 05:49:57  krisrose
%% Attributes next...
%%
%% Revision 3.5  2013/09/18 14:42:43  krisrose
%% Moved 'dragon' to 'gentle'.
%%
%% Revision 3.4  2013/09/12 05:20:53  krisrose
%% Making dragon example stand-alone.
%%
%% Revision 3.3  2013/09/12 01:37:09  krisrose
%% Regular expression updates for Martin.
%% PassLocationInformation creation fix.
%%
%% Revision 3.2  2013/09/01 05:13:00  krisrose
%% $[Get] fixed.
%%
%% Revision 3.1  2013/08/23 07:12:32  krisrose
%% Manual started.
%% Sharing of variables avoided.
%%
%% Revision 1.9  2013/07/26 21:39:15  krisrose
%% Moved large (mostly obsolete) samples collection here.
%%
%% Revision 1.8  2013/06/26 05:54:08  krisrose
%% Writing...
%%
%% Revision 1.7  2013/06/25 09:43:54  krisrose
%% Bibliography updates.
%% Raw terms extracted to appendix (and trying .-binders again...)
%%
%% Revision 1.6  2013/06/22 23:53:18  krisrose
%% CVS Ids in foornotes.
%%
%% Revision 1.5  2013/06/22 23:40:20  krisrose
%% Draft marker added.
%%
%% Revision 1.4  2013/06/22 19:22:19  krisrose
%% Over.
%%
%% Revision 1.3  2013/06/21 11:58:51  krisrose
%% HOACS notation and sorts coming along...
%%
%% Revision 1.2  2013/06/20 17:31:04  krisrose
%% #s eliminated...
%%
%% Revision 1.1  2013/06/11 14:51:37  krisrose
%% All documentation collected in one project (to permit more nimble development of crsx and hoacs).
%%
%% Revision 1.1  2013/06/11 14:11:29  krisrose
%% HOACS taking shape.
%%
%% Revision 3.2  2013/06/07 00:58:28  krisrose
%% Sharing implementation in progress...do not use!
%%
%% Revision 3.1  2013/05/31 21:01:48  krisrose
%% Started hoacs parser.
%% Mutation substitution coded by disabled except for debugging.
%%
%% New.

%%---------------------------------------------------------------------
% Tell Emacs that this is a LaTeX document and how it is formatted:
% Local Variables:
% mode:latex
% fill-column:100
% End:
